\begin{RecordType}{\hyperB{IT::SequentialCoupling}{SequentialCoupling}}{\Alink{IT::Problem}{Problem}}{}{}{Record with data for a general sequential coupling.
}
\KeyItem{\hyperB{SequentialCoupling::TYPE}{TYPE}} <<Missing formatter for <class 'ist.nodes.Selection'>>>
\KeyItem{\hyperB{SequentialCoupling::description}{description}}{String (generic)}{\textlangle{\it optional }\textrangle}{}{Short description of the solved problem.
Is displayed in the main log, and possibly in other text output files.}
\KeyItem{\hyperB{SequentialCoupling::mesh}{mesh}}{record: \Alink{IT::Mesh}{Mesh}}{\textlangle{\it obligatory }\textrangle}{}{Computational mesh common to all equations.}
\KeyItem{\hyperB{SequentialCoupling::time}{time}}{record: \Alink{IT::TimeGovernor}{TimeGovernor}}{\textlangle{\it optional }\textrangle}{}{Simulation time frame and time step.}
\KeyItem{\hyperB{SequentialCoupling::primary-equation}{primary\_equation}}{abstract type: \Alink{IT::DarcyFlowMH}{DarcyFlowMH}}{\textlangle{\it obligatory }\textrangle}{}{Primary equation, have all data given.}
\KeyItem{\hyperB{SequentialCoupling::secondary-equation}{secondary\_equation}}{abstract type: \Alink{IT::Transport}{Transport}}{\textlangle{\it optional }\textrangle}{}{The equation that depends (the velocity field) on the result of the primary equation.}
\end{RecordType}